\documentclass[letterpaper,twocolumn,10pt]{article}
\usepackage[margin=1.0in]{geometry}
\usepackage{tikz}
\usepackage{amsmath}
\usepackage{filecontents}
\usepackage{graphicx}

\begin{document}
\date{}
\onecolumn

\title{\Large \bf ECE 141: Final Project \\
Steven Chu (905-094-800) : Section 1A}

\maketitle

\section*{Introduction}
The goal of this project is to model the dynamics of COVID-19 spread through a given set of differential equations and to do analysis upon how a controller can be implemented to stabilize the rate of change in infected individuals. \\
We will use a population model with the following five classes:\hspace*{\fill}
\vspace{3mm}
\linebreak
S: number of susceptible individuals\\
E: number of exposed individuals (they show no symptoms and are not contagious)\\
I: number of infected individuals with mild or no symptoms at all (it is believed that about\\
80\% of the infected population belongs to this class)\\
J: number of infected individuals that are seriously ill\\
R: number of recovered individuals\\
D: number of individuals that have died\hspace*{\fill}
\linebreak

\begin{equation*}
\begin{split}
\dot{S} &= -\beta_{1}SI - \beta_{2}SJ\\
\dot{E} &= \beta_{1}SI + \beta_{2}SJ - \gamma E\\
\dot{I} &= \sigma_{1}\gamma E - \rho_{1}I\\
\dot{J} &= \sigma_{2}\gamma E - \rho_{2}J - qJ\\
\dot{R} &= \rho_{1}I + \rho_{2}J\\
\dot{D} &= qJ
\end{split}
\end{equation*}

Using the following values for the parameters:
\[\beta_{1} = 0.25, \beta_{2} = 0.025, \gamma = 0.2, \sigma_{1} = 0.8, \sigma_{2} = 0.2, \rho_{1} = \frac{1}{14}, \rho_{2} = \frac{1}{21}, q = 0.15\]

\section*{Problem 1}
We will simulate the system specified in the introduction using the initial conditions:
\[S(0) = 0.8, E(0) = 0.145, I(0) = 0.05, J(0) = 0.005, R(0) = 0, D(0) = 0\]
Shown in the figure, we simulate the system of equations in Simulink. From the given initial conditions, the model shows the time evolution of all six parameters. We find that, the susceptible portion of the population (S, yellow), drops steadily, ultimately dropping to about 5\% of the population. In contrast, the recovered portion of the population (R, purple), steadily rises, stabilizing at nearly 80\%. The exposed portion of the population (E, blue), gradually converges to 0\%. The infected portion of the population (I, orange), rises, and reaches its peak of just over 20\% of the population, but then gradually converges to 0\%. The dead portion of the population (D, teal), steadily rises, and converges to about 14\%. Finally, the seriously ill portion of the population (J, green), rises, peaking at just over 2\% of the population, and eventually converges to 0\%.

\begin{figure}[t!]
\centering
\includegraphics[width= 130mm]{Model_1.png}
\caption{Time Evolution of initially presented model \label{fig:Model_1}}
\end{figure}

\section*{Problem 2}
Comment on the number of beds needed to treat the portion of the population in class J: It seems that the size of class J rapidly peaks at just over 2\% of the total population early on in the course of the model, as shown in the figure. It then drops to converge to zero. It seems like the number of hospital beds needed to treat these individuals would spike around that time, and the demand would likely overwhelm the supply at that point in time.

\begin{figure}[b!]
\centering
\includegraphics[width= 100mm]{Proportion_ClassJ.png}
\caption{Time Evolution of the Class J Component \label{fig:Proportion_ClassJ}}
\end{figure}

\section*{Problem 3}
Now, we modify the initial model to consider the effect of using masks. We introduce an input \(u\) to the original model:

\begin{equation*}
\begin{split}
\dot{S} &= -\beta_{1}SI - \beta_{2}SJ\\
\dot{E} &= \beta_{1}SI + \beta_{2}SJ - (1-u)\gamma E\\
\dot{I} &= (1-u)\sigma_{1}\gamma E - \rho_{1}I\\
\dot{J} &= (1-u)\sigma_{2}\gamma E - \rho_{2}J - qJ\\
\dot{R} &= \rho_{1}I + \rho_{2}J\\
\dot{D} &= qJ
\end{split}
\end{equation*}

In order to compute the equilibrium pairs for this system of differential equations, we first write them in normal form:

\begin{equation*}
\begin{split}
\dot{x_{1}} &= -\beta_{1}x_{1}x_{3} - \beta_{2}x_{1}x_{4}\\
\dot{x_{2}} &= \beta_{1}x_{1}x_{3} + \beta_{2}x_{1}x_{4} - (1-u)\gamma x_{2}\\
\dot{x_{3}} &= (1-u)\sigma_{1}\gamma x_{2} - \rho_{1}x_{3}\\
\dot{x_{4}} &= (1-u)\sigma_{2}\gamma x_{2} - \rho_{2}x_{4} - qx_{4}\\
\dot{x_{5}} &= \rho_{1}x_{3} + \rho_{2}x_{4}\\
\dot{x_{6}} &= qx_{4}
\end{split}
\end{equation*}
To compute the equilibrium pair, we want to find \((x_{e}, u_{e})\) that is a solution to the normal form equation \(\dot{x} = f(x_{e}, u_{e}) = 0\) where \(x\) is the vector of classes \((x_{1}, x_{2}, ..., x_{6})\) and \(f(x, u)\) is the vector containing the system of equations described above.\\
Ultimately we want to come up with a linear model of this system in the form of: \[\frac{d}{dt}\delta x = \frac{df}{dx}\bigg|_{(x_{e}, u_{e})}\delta x + \frac{df}{du}\bigg|_{(x_{e}, u_{e})}\delta u \rightarrow \frac{d}{dt}\delta x = A\delta x + B\delta u\]
Where A and B are matrices of constants. \hspace*{\fill}
\vspace{2mm}
\linebreak
First, we see that:
\begin{equation*}
f(x, u)
=
\begin{bmatrix}
-\beta_{1}x_{1}x_{3} - \beta_{2}x_{1}x_{4}\\
\beta_{1}x_{1}x_{3} + \beta_{2}x_{1}x_{4} - (1-u)\gamma x_{2}\\
(1-u)\sigma_{1}\gamma x_{2} - \rho_{1}x_{3}\\
(1-u)\sigma_{2}\gamma x_{2} - \rho_{2}x_{4} - qx_{4}\\
\rho_{1}x_{3} + \rho_{2}x_{4}\\
qx_{4}
\end{bmatrix}
\end{equation*}
We then calculate that:
\begin{equation*}
\frac{df(x, u)}{dx}
=
\begin{bmatrix}
-\beta_{1}x_{3} - \beta_{2}x_{4} & 0 & -\beta_{1}x_{1} - \beta_{2}x_{1}x_{4}  & -\beta_{1}x_{1}x_{3} - \beta_{2}x_{1} & 0 & 0\\
\beta_{1}x_{3} + \beta_{2}x_{4} & -(1-u)\gamma & \beta_{1}x_{1} + \beta_{2}x_{1}x_{4} & \beta_{1}x_{1}x_{3} + \beta_{2}x_{1} & 0 & 0\\
0 & (1-u)\sigma_{1}\gamma & -\rho_{1} & 0 & 0 & 0\\
0 & (1-u)\sigma_{2}\gamma & 0 & -\rho_{2}-q & 0 & 0\\
0 & 0 & \rho_{1} & \rho_{2} & 0 & 0\\
0 & 0 & 0 & q & 0 & 0
\end{bmatrix}
\end{equation*}
\vspace{28mm}

And also that:

\begin{equation*}
\frac{df(x, u)}{du}
=
\begin{bmatrix}
0\\
\gamma x_{2}\\
-\sigma_{1}\gamma x_{2}\\
-\sigma_{2}\gamma x_{2}\\
0\\
0
\end{bmatrix}
\end{equation*}

Now, we solve the system of equations:
\begin{equation*}
f(x_{e}, u_{e})
=
\begin{bmatrix}
-\beta_{1}x_{1}x_{3} - \beta_{2}x_{1}x_{4}\\
\beta_{1}x_{1}x_{3} + \beta_{2}x_{1}x_{4} - (1-u)\gamma x_{2}\\
(1-u)\sigma_{1}\gamma x_{2} - \rho_{1}x_{3}\\
(1-u)\sigma_{2}\gamma x_{2} - \rho_{2}x_{4} - qx_{4}\\
\rho_{1}x_{3} + \rho_{2}x_{4}\\
qx_{4}
\end{bmatrix}
=
\begin{bmatrix}
0\\
0\\
0\\
0\\
0\\
0
\end{bmatrix}
\end{equation*}

Upon inspection, we find that an equilibrium pair that is a solution to the above system is:
\begin{equation*}
x_{e}
=
\begin{bmatrix}
1\\
0\\
0\\
0\\
0\\
0\\
\end{bmatrix}
\end{equation*}

Where \(u_e\) can take on any value in the range \(0 \leq u < 1\). Let it also be noted that any distribution of the proportion of the population between \(x_{1}, x_{5}, x_{6}\) (corresponding to classes S, R, and D) is an equilibrium pair, provided that the other three classes all take on the value of zero. \hspace*{\fill}
\vspace{3mm}
\linebreak
When we look closer at what the values of these equilibrium pairs entail, it is clear that they do not provide any useful insight. It is obvious that distributing the entire population among the classes of S, R, and D would result in stability, because that means the disease either does not exist, or has run its course completely. If there are no individuals infected, seriously ill, or exposed to the disease, then that simply means the disease is no longer active or just has not happened (which is what the example equilibrium pair above shows).

\section*{Problem 4}
To get the transfer function from J(0) to J, we first need to compute the result of the linear model for \(x_{4}\).
From the matrices computed in Problem 3, and the example equilibrium pair, we can see that:
\[\frac{d}{dt}\delta x_{4} = (1-u)\sigma_{2}\gamma\delta x_{2} + (-\rho_{2} - q)\delta x_{4} \rightarrow \frac{d}{dt}\delta J = (1-u)\sigma_{2}\gamma\delta E + (-\rho_{2} - q)\delta J\]
Taking the Laplace Transform:
\[sJ(s) - J(0) = (1-u)\sigma_{2}\gamma E(s) + (-\rho_{2} - q)J(s)\]
\[sJ(s) + (\rho_{2} + q)J(s) = (1-u)\sigma_{2}\gamma E(s) + J(0)\]
We would like to express \(E(s)\) in terms of \(J(s)\):
\[\dot{I} = (1-u)\sigma_{1}\gamma E - \rho_{1}I\]
Taking the Laplace Transform:
\[sI(s) = (1-u)\sigma_{1}\gamma E(s) - \rho_{1}I(s)\]
\[I(s) = \frac{(1-u)\sigma_{1}\gamma E(s)}{s+\rho_{1}}\]
Now, we use this to find an expression for \(E(s)\):
\[\dot{E} = \beta_{1}I + \beta_{2}J - (1-u)\gamma E\]
Taking the Laplace Transform:
\[sE(s) = \beta_{1}I(s) + \beta_{2}J(s) - (1-u)\gamma E(s)\]
Using our derived expression for \(I(s)\):
\[E(s)(s+(1-u)\gamma) = \frac{\beta_{1}(1-u)\sigma_{1}\gamma E(s)}{s+\rho_{1}} + \beta_{2}J(s)\]
\[E(s) = \frac{\beta_{2}J(s)}{s - \frac{\beta_{1}(1-u)\sigma_{1}\gamma}{s+\rho_{1}}+(1-u)\gamma}\]
Finally, we substitute this expression for \(E(s)\) back into our original equation:
\[J(s)\left((s+\rho_{2}+q)-(1-u)\sigma_{2}\gamma\left(\frac{\beta_{2}}{s - \frac{\beta_{1}(1-u)\sigma_{1}\gamma}{s+\rho_{1}}+(1-u)\gamma}\right)\right) = J(0)\]
We get the overall transfer function:
\[\boxed{\frac{J}{J_{0}}=\frac{s(s+\rho_{1}) + (1-u)\gamma(s+\rho_{1}}{(s+\rho_{2}+q)(s(s+\rho_{1})-(1-u)\gamma(s+\rho_{1}))}}\]
From this transfer function, we can see that the poles will become more negative the more \(u\) approaches 1, and more positive the more \(u\) approaches 0.\hspace*{\fill}
\vspace{2mm}
\linebreak

\section*{Problem 5}
To simulate the linearized model, we must first compute the rest of the A and B matrices to obtain the complete system of equations. For this instance of the linearized model, we will use the example equilibrium pair given in Problem 3:
\begin{equation*}
A
=
\frac{df(x_{e}, u_{e})}{dx}
=
\begin{bmatrix}
0 & 0 & -\beta_{1} & -\beta_{2} & 0 & 0\\
0 & -(1 - u)\gamma & \beta_{1} & \beta_{2} & 0 & 0\\
0 & (1-u)\sigma_{1}\gamma & -\rho_{1} & 0 & 0 & 0\\
0 & (1-u)\sigma_{2}\gamma & 0 & -\rho_{2}-q & 0 & 0\\
0 & 0 & \rho_{1} & \rho_{2} & 0 & 0\\
0 & 0 & 0 & q & 0 & 0
\end{bmatrix}
\hspace{7mm}
B
=
\frac{df(x_{e}, u_{e})}{du}
=
\begin{bmatrix}
0\\
0\\
0\\
0\\
0\\
0
\end{bmatrix}
\end{equation*}
Overall the linearized system comes out to look something like:
\begin{equation*}
\frac{d}{dt}\delta x
=
\begin{bmatrix}
-\beta_{1}\delta x_{3} - \beta_{2}\delta x_{4}\\
-(1-u)\gamma\delta x_{2} + \beta_{1}\delta x_{3} + \beta_{2}\delta x_{4}\\
(1-u)\sigma_{1}\gamma\delta x_{2} - \rho_{1}\delta x_{3}\\
(1-u)\sigma_{2}\gamma\delta x_{2} -\rho_{2}\delta x_{4} - q\delta x_{4}\\
\rho_{1}\delta x_{3} + \rho_{2}\delta x_{4}\\
q\delta x_{4}
\end{bmatrix}
=
\begin{bmatrix}
-\beta_{1}\delta I - \beta_{2}\delta J\\
-(1-u)\gamma\delta E + \beta_{1}\delta I + \beta_{2}\delta J\\
(1-u)\sigma_{1}\gamma\delta E - \rho_{1}\delta I\\
(1-u)\sigma_{2}\gamma\delta E -\rho_{2}\delta J - q\delta J\\
\rho_{1}\delta I + \rho_{2}\delta J\\
q\delta J
\end{bmatrix}
\end{equation*}
\vspace{3mm}
\linebreak
Simulating this linear model in Simulink yields the time evolution plot shown in the figure below:
\begin{figure}[htb!]
\centering
\includegraphics[width= 130mm]{Model_5.png}
\caption{Time Evolution of linearized model (u = 0.5) \label{fig:Model_5}}
\end{figure}

From this plot, we can gather that this linearized model is successful in simulating initial rates of change in the various population classes, but does not model the subsequent time evolution completely accurately. It can also be seen that this model is not stable, and the parameters grow past the standard proportions and diverge after a certain period of time.\hspace*{\fill}
\vspace{2mm}
\linebreak
However it can be seen that by varying u, this model correctly depicts the logical change in the rate of infection (the higher the proportion u is, the slower the spread of the disease progresses). This is shown in the figures below:

\begin{figure}[htb!]
\centering
\includegraphics[width= 130mm]{Model_5_high.png}
\caption{Time Evolution of linearized model (u = 0.15) \label{fig:Model_5}}
\end{figure}

\begin{figure}[htb!]
\centering
\includegraphics[width= 130mm]{Model_5_low.png}
\caption{Time Evolution of linearized model (u = 0.85) \label{fig:Model_5}}
\end{figure}

\section*{Problem 6}
To implement this switched controller, the unit step source block can be used in Simulink. Rather than feeding in a constant in the place of input u, a staircase function can be implemented using this block and used as an input instead.\hspace*{\fill}
\vspace{2mm}
\linebreak
Upon experimentation, the function:
\[u(t) = y=10[u(x)*u(65-x)]+70[u(x)*u(90-x)]\]
was found to successfully contain the proportion of the population in class J to under 1\% at all times.\hspace*{\fill}
\linebreak

It has a plot as shown in the figure below:
\begin{figure}[htb!]
\centering
\includegraphics[width= 130mm]{Controller_u.png}
\caption{Plot of parameter u switched values \label{fig:Controller_u}}
\end{figure}

As is shown in the plot, the switching instants are comprised of three phases: First, from time = 0 \(\rightarrow\) 65, 80\% of the population must wear masks. Second, from time = 65 \(\rightarrow\) 90, this percentage is reduced to 70\%. Lastly, for time \(>\) 90, masks can be removed from the model entirely, with none of the population wearing masks. \hspace*{\fill}
\vspace{3mm}
\linebreak

When examining the effect on the time evolution of class J, it can be seen that the initial peak of 2\% in the initial model is lowered to about 0.95\% by the implementation of the initial proportion of population wearing masks. As would be expected however, the proportion of class J spikes up and peaks again shortly after both instances where the proportion input \(u\) is lowered. However, with the values of \(u\) chosen, these peaks are similar in magnitude to the initial peak, and do not surpass 1\%. As in the initial model, after these peaks the proportion of class J converges to zero. This is shown in the figure below:

\begin{figure}[htb!]
\centering
\includegraphics[width= 130mm]{Proportion_J_Controlled.png}
\caption{Time Evolution of Class J Component with Controller \label{fig:Proportion_J_Controlled}}
\end{figure}
\vspace{25mm}
Overall, it seems that although the controller successfully contained the proportion of the population in class \(J\) at any given time to less than 1\%, the overall proportion of the population that ends in classes \(R\) and \(D\) remains relatively the same. Class \(R\) (purple) still steadily rises, eventually converging to 80\%, while Class \(D\) (teal) also rises and stabilizes at around 14\%. This is shown in the figure below:

\begin{figure}[htb!]
\centering
\includegraphics[width= 130mm]{Model_Controlled.png}
\caption{Time Evolution of overall model with controller \label{fig:Model_Controlled}}
\end{figure}
\vspace{85mm}
\section*{Problem 7}
Unfortunately, this control function does not work when the initial conditions are varied significantly. \\
For example, initial conditions of \(S(0) = 0.9, E(0) = 0.045, I(0) = 0.05, J(0.005), R(0) = 0, J(0) = 0\) yield the plot:

\begin{figure}[htb!]
\centering
\includegraphics[width= 115mm]{Proportion_J_init1.png}
\caption{Time Evolution of Class J, Initial Condition Variation 1 \label{fig:Proportion_J_init1}}
\end{figure}

Here we can see that the input \(u\) was lowered too soon, causing the later peaks to rise above 1\%.
Initial conditions of \(S(0) = 0.9, E(0) = 0.045, I(0) = 0.05, J(0.005), R(0) = 0, J(0) = 0\) yield the plot:

\begin{figure}[htb!]
\centering
\includegraphics[width= 115mm]{Proportion_J_init2.png}
\caption{Time Evolution of Class J, Initial Condition Variation 2 \label{fig:Proportion_J_init2}}
\end{figure}
\vspace{2mm}
Here we can see that the initial proportion of \(u\) was not high enough, and Class J exceeded 1\% on the initial peak.\hspace*{\fill}
\vspace{3mm}
\linebreak
It is unlikely that the controller could work for any initial condition (e.g. conditions where the initial condition for Class J exceeds 1\%), however, pushing the input \(u\) as high as possible and continuing it until time \(>\) 100 would account for the widest possible range of initial conditions.

\section*{Simulink Model Implementations}
\begin{figure}[htb!]
\centering
\includegraphics[width= 170mm]{Simulink_Problem_1.png}
\caption{Model for Nonlinear System (Problem 1)\label{fig:Simulink_Problem_1}}
\end{figure}

\begin{figure}[htb!]
\centering
\includegraphics[width= 170mm]{Simulink_Problem_5.png}
\caption{Model for Nonlinear System (Problem 5)\label{fig:Simulink_Problem_5}}
\end{figure}

\end{document}